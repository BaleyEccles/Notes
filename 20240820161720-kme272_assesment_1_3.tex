% Created 2024-08-24 Sat 14:46
% Intended LaTeX compiler: pdflatex
\documentclass[11pt]{article}
\usepackage[utf8]{inputenc}
\usepackage[T1]{fontenc}
\usepackage{graphicx}
\usepackage{longtable}
\usepackage{wrapfig}
\usepackage{rotating}
\usepackage[normalem]{ulem}
\usepackage{amsmath}
\usepackage{amssymb}
\usepackage{capt-of}
\usepackage{hyperref}
\usepackage{minted}
\usepackage[a4paper, margin=2.5cm]{geometry}
\usepackage{minted}
\usepackage{fontspec}
\setmonofont{Iosevka}
\setminted{fontsize=\small, frame=single, breaklines=true}
\usemintedstyle{emacs}
\author{Baley Eccles - 652137}
\date{\textit{{[}2024-08-20 Tue 16:17]}}
\title{KME272 - Assesment 1.3}
\hypersetup{
 pdfauthor={Baley Eccles - 652137},
 pdftitle={KME272 - Assesment 1.3},
 pdfkeywords={},
 pdfsubject={},
 pdfcreator={Emacs 29.4 (Org mode 9.8)}, 
 pdflang={English}}
\begin{document}

\maketitle
\tableofcontents

\section{KME272 - Assesment 1.3}
\label{sec:org4569aa7}
\subsection{1}
\label{sec:orgc39f13e}
\begin{minted}[]{octave}
A = [1   , -1/4, 0   , 0   , 0   , 0   ; ...
     -1/4, 1   , -1/4, 0   , -1/4, 0   ; ...
     0   , -1/4, 1   , -1/4, 0   , 0   ; ...
     0   , -1/4, 0   , 1   , -1/4, 0   ; ...
     0   , 0   , -1/4, -1/4, 1   , -1/4; ...
     0   , 0   , 0   , 0   , -1/4, 1   ;];
b = [20;0;20;5/4;5/4;85/4];
[M,N] = size(A);
if M~=N
  error('A must be a square matrix')
end

rowtest = abs(diag(A)) > sum(abs(A-diag(diag(A))),2);
index = find(rowtest==0);
if ~isempty(index)
  disp("The matrix is not diagonally dominant.")
  disp("Take the results with caution.")
  pause(2)
end

x = [0; 0; 0; 0; 0; 0];
x0 = x;

reldiff_tolerance = 1e-3;
reldiff = 100;
k = 0;

fprintf('CASE 1: Diagonally Dominant\n----------------------------\nJacobi iteration\n')
while reldiff > reldiff_tolerance
  x0(1) = (b(1)-A(1,2:M)*x(2:M))/A(1,1);
  for j = 2 : M-1
    termsl = A(j,1:j-1)*x(1:j-1);
    termsr = A(j,j+1:M)*x(j+1:M);
    x0(j) = (b(j)-termsl-termsr)/A(j,j);
  end
  x0(M) = (b(M)-A(M,1:M-1)*x(1:M-1))/A(M,M);
  reldiff = abs(norm(x-x0))/norm(x0);
  k = k+1;
  residual = norm(x - x0);
  x = x0;
  if k == 10
    x10 = x;
  end

  fprintf('%d, [%.2f, %.2f, %.2f, %.2f, %.2f, %.2f], residual = %.2f\n',k,x(1),x(2),x(3),x(4),x(5),x(6), residual);
  x_J=x;
end
x_J = x;
k_J = k;

%% Gauss Seidel Method
fprintf('----------------------------\nGauss-Seidel iteration\n')
x = [0; 0; 0; 0; 0; 0];
k_G = 0;
reldiff = 100;
while reldiff > reldiff_tolerance
  x0 = x;
  for j = 1:size(A,1)
    x(j) = (b(j) - sum(A(j,:)'.*x) + A(j,j)*x(j)) / A(j,j);
  end
  reldiff = abs(norm(x-x0))/norm(x0);
  k = k+1;
  residual = norm(x-x0);
  fprintf('%d, [%.2f, %.2f, %.2f, %.2f, %.2f, %.2f], residual = %.2f\n',k,x(1),x(2),x(3),x(4),x(5),x(6), residual);

end
x_G=x;
k_G=k;
fprintf('----------------------------\nSummary\n')
fprintf('Using Jacobi iteration, [t1, t2, t3, t4, t5, t6] = [%.2f, %.2f, %.2f, %.2f, %.2f, %.2f] after %i %.2f\n',k,x_J(1),x_J(2),x_J(3),x_J(4),x_J(5),x_J(6),k_J);
fprintf('Using Gauss-Seidel iteration, [t1, t2, t3, t4, t5, t6] = [%.2f, %.2f, %.2f, %.2f, %.2f, %.2f] after %i %.2f\n',k,x_G(1),x_G(2),x_G(3),x_G(4),x_G(5),x_G(6),k_G);
\end{minted}

\phantomsection
\label{}
\begin{verbatim}
CASE 1: Diagonally Dominant
----------------------------
Jacobi iteration
1, [20.00, 0.00, 20.00, 1.25, 1.25, 21.25], residual = 35.42
2, [20.00, 10.31, 20.31, 1.56, 11.88, 21.56], residual = 14.82
3, [22.58, 13.05, 22.97, 6.80, 12.11, 24.22], residual = 7.46
4, [23.26, 14.41, 24.96, 7.54, 14.75, 24.28], residual = 3.72
5, [23.60, 15.74, 25.49, 8.54, 15.44, 24.94], residual = 2.02
6, [23.94, 16.13, 26.07, 9.05, 15.99, 25.11], residual = 1.09
7, [24.03, 16.50, 26.30, 9.28, 16.31, 25.25], residual = 0.61
8, [24.12, 16.66, 26.45, 9.45, 16.46, 25.33], residual = 0.34
9, [24.16, 16.76, 26.53, 9.53, 16.56, 25.36], residual = 0.19
10, [24.19, 16.81, 26.57, 9.58, 16.61, 25.39], residual = 0.11
11, [24.20, 16.84, 26.60, 9.60, 16.63, 25.40], residual = 0.06
12, [24.21, 16.86, 26.61, 9.62, 16.65, 25.41], residual = 0.03
----------------------------
Gauss-Seidel iteration
13, [20.00, 5.00, 21.25, 2.50, 7.19, 23.05], residual = 38.28
14, [21.25, 12.42, 23.73, 6.15, 14.48, 24.87], residual = 11.52
15, [23.11, 15.33, 25.37, 8.70, 15.99, 25.25], residual = 4.85
16, [23.83, 16.30, 26.25, 9.32, 16.45, 25.36], residual = 1.69
17, [24.07, 16.69, 26.50, 9.54, 16.60, 25.40], residual = 0.59
18, [24.17, 16.82, 26.59, 9.61, 16.65, 25.41], residual = 0.20
19, [24.20, 16.86, 26.62, 9.63, 16.66, 25.42], residual = 0.06
20, [24.22, 16.87, 26.63, 9.63, 16.67, 25.42], residual = 0.02
----------------------------
Summary
Using Jacobi iteration, [t1, t2, t3, t4, t5, t6] = [20.00, 24.21, 16.86, 26.61, 9.62, 16.65] after 25.4086 12.00
Using Gauss-Seidel iteration, [t1, t2, t3, t4, t5, t6] = [20.00, 24.22, 16.87, 26.63, 9.63, 16.67] after 25.4172 20.00
\end{verbatim}
\end{document}
