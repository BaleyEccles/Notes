% Created 2024-08-29 Thu 12:51
% Intended LaTeX compiler: pdflatex
\documentclass[11pt]{article}
\usepackage[utf8]{inputenc}
\usepackage[T1]{fontenc}
\usepackage{graphicx}
\usepackage{longtable}
\usepackage{wrapfig}
\usepackage{rotating}
\usepackage[normalem]{ulem}
\usepackage{amsmath}
\usepackage{amssymb}
\usepackage{capt-of}
\usepackage{hyperref}
\usepackage{minted}
\usepackage[a4paper, margin=2.5cm]{geometry}
\usepackage{minted}
\usepackage{fontspec}
\setmonofont{Iosevka}
\setminted{fontsize=\small, frame=single, breaklines=true}
\usemintedstyle{emacs}
\author{Baley Eccles - 652137}
\date{\textit{{[}2024-08-26 Mon 12:08]}}
\title{ENG204 - Signals and Linear Systems – Assignment 1.2}
\hypersetup{
 pdfauthor={Baley Eccles - 652137},
 pdftitle={ENG204 - Signals and Linear Systems – Assignment 1.2},
 pdfkeywords={},
 pdfsubject={},
 pdfcreator={Emacs 29.4 (Org mode 9.8)}, 
 pdflang={English}}
\begin{document}

\maketitle
\tableofcontents

\section{ENG204 - Signals and Linear Systems – Assignment 1.2}
\label{sec:org7409fdb}
\subsection{b}
\label{sec:orgf62f837}
\begin{minted}[]{octave}
clc
clear
pkg load symbolic
syms mt at ms as  ...
     Fkt Fks Fcs Fks
                                % sum F_y = ma
eq1 = mt*at == Fkt - Fks - Fcs;
eq2 = ms*as == Fks + Fcs;

                                % Sub in force equations
syms kt u zt zs dzs dzt ks cs
FktEqu = -kt*(u-zt);
FksEqu = -ks*(zt-zs);
FcsEqu = -cs*(dzt-dzs);
eq1=subs(subs(subs(eq1,Fkt,FktEqu),Fks,FksEqu),Fcs,FcsEqu);
eq2=subs(subs(subs(eq2,Fkt,FktEqu),Fks,FksEqu),Fcs,FcsEqu);
\end{minted}
\subsubsection{Discrete}
\label{sec:org6520ed6}
\begin{minted}[]{octave}
                                % Sub in forward difference
syms zs zs1 zs2 zt zt1 zt2 Ts vs1 vt1 vs vt
% First forward difference of displacement
dzsEqu = (zs1-zs)/(Ts);
dztEqu = (zt1-zt)/(Ts);
% First forward differences of acceleration
dvsEqu = (vs1-vs)/(Ts);
dvtEqu = (vt1-vt)/(Ts);
eq1=subs(eq1,at,dvtEqu);
eq1=subs(eq1,dzs,dzsEqu);
eq1=simplify(subs(eq1,dzt,dztEqu))
eq2=subs(eq2,as,dvsEqu);
eq2=subs(eq2,dzs,dzsEqu);
eq2=simplify(subs(eq2,dzt,dztEqu))

\end{minted}
\begin{enumerate}
\item Difference Equations
\label{sec:org53c5338}
\begin{itemize}
\item \(\frac{m_t \left(- v_s[n+1] + v_s[n+1]\right)}{T_s} = \frac{T_s \left(k_s \left(z_s[n] - z_t[n]\right) + kt \left(u[n] - z_t[n]\right)\right) - c_s \left(z_s[n] - z_s[n+1] - z_t[n] + z_t[n+1]\right)}{T_s}\)
\item \(\frac{m_s \left(- v_t[n] + v_t[n+1[\right)}{T_s} = \frac{- T_s k_s \left(z_s[n] - z_t[n]\right) + c_s \left(z_s - z_s[n+1] - z_t[n] + z_t[n+1]\right)}{T_s}\)
\end{itemize}
\end{enumerate}
\subsubsection{Continious}
\label{sec:org1ddf125}
\begin{minted}[]{octave}

fprintf("at = \n")
latex(expand(solve([eq1,eq2],at)))
fprintf("as = \n")
latex(expand(solve([eq1,eq2],as)))

\end{minted}
\begin{enumerate}
\item ODE
\label{sec:org6cebd75}
\begin{itemize}
\item \(a_t=-\frac{c_s \dot{z}_s}{m_t} + \frac{c_s \dot{z}_t}{m_t} - \frac{k_s z_s}{m_t} + \frac{k_s z_t}{m_t} - \frac{k_t u}{m_t} + \frac{k_t z_t}{m_t}\)
\item \(a_s=\frac{c_s \dot{z}_s}{m_s} - \frac{c_s \dot{z}_t}{m_s} + \frac{k_s z_s}{m_s} - \frac{k_s z_t}{m_s}\)
\item \(v_t=\dot{z}_t\)
\item \(v_s=\dot{z}_s\)
\end{itemize}
\end{enumerate}
\subsection{c}
\label{sec:orgca15acf}


Using:
\begin{itemize}
\item \(\underline{q}[n+1]=\underline{A}q[n]+\underline{b}x[n]\)
\item \(y[n]=\underline{C}q[n]+dx[n]\)
\end{itemize}
We chose the following state varibles:

\begin{itemize}
\item \(q_1[n] = z_t[n]\)
\item \(q_2[n] = v_t[n]\)
\item \(q_3[n] = z_s[n]\)
\item \(q_4[n] = v_s[n]\)
\item \(x[n] = u[n]\)
\item \(y[n] = \begin{bmatrix} z_t[n] \\ z_s[n] \end{bmatrix}\)
\end{itemize}
\subsubsection{Continious}
\label{sec:org0865f7d}
\begin{enumerate}
\item State Equation
\label{sec:org2c185c4}
\begin{minted}[]{octave}
syms q1n q1n1 q2n q2n1 q3n q3n1 q4n q4n1

eq1 = subs(eq1,zt,q1n);
eq1 = subs(eq1,zs,q3n);


eq1 = subs(eq1,dzt,q1n1);
eq1 = subs(eq1,at,q2n1);
eq1 = subs(eq1,dzs,q3n1);
eq1 = subs(eq1,as,q4n1);

eq2 = subs(eq2,zt,q1n);
eq2 = subs(eq2,zs,q3n);

eq2 = subs(eq2,dzt,q1n1);
eq2 = subs(eq2,at,q2n1);
eq2 = subs(eq2,dzs,q3n1);
eq2 = subs(eq2,as,q4n1);

eq1 = subs(eq1, q1n1, eqq1n1);
eq1 = subs(eq1, q3n1, eqq3n1);
eq2 = subs(eq2, q1n1, eqq1n1);
eq2 = subs(eq2, q3n1, eqq3n1);
expand(simplify(solve([eq1,eq2],q2n1)));
expand(simplify(solve([eq1,eq2],q4n1)));

\end{minted}


\begin{itemize}
\item \(q_1[n+1]=q_2[n]\)
\item \(q_2[n+1]=\frac{c_s q_2[n]}{m_t} - \frac{c_s q_4[n]}{m_t} + \frac{k_s q_1[n]}{m_t} - \frac{k_s q_3[n]}{m_t} + \frac{k_t q_1[n]}{m_t} - \frac{k_t u[n]}{m_t}\)
\item \(q_3[n+3]=q_4[n]\)
\item \(q_4[n+1]=-\frac{c_s q_2[n]}{m_s} + \frac{c_s q_4[n]}{m_s} - \frac{k_s q_1[n]}{m_s} + \frac{k_s q_3[n]}{m_s}\)
\end{itemize}

\[\underline{A} = \begin{bmatrix}
0 & 1 & 0 & 0 \\
\frac{k_s}{m_t}+ \frac{k_t }{m_t} & \frac{c_s }{m_t} &  - \frac{k_s }{m_t} &  - \frac{c_s }{m_t} \\
0 & 0 & 0 & 1 \\
-\frac{k_s }{m_s} & -\frac{c_s }{m_s} & \frac{k_s }{m_s} & \frac{c_s }{m_s}
\end{bmatrix}\]

\[\underline{B} = \begin{bmatrix}
0 \\
-\frac{k_t}{m_t} \\
0 \\
0
\end{bmatrix}\]
\item Output Equation
\label{sec:org566e23b}
Using
\begin{itemize}
\item \(z_t=q_1\)
\item \(z_s=q_3\)
\end{itemize}

\[\underline{C} = \begin{bmatrix}
1 & 0 & 0 & 0 \\
0 & 0 & 1 & 0
\end{bmatrix}\]

\[d = \begin{bmatrix}
0 \\
0
\end{bmatrix}\]
\end{enumerate}
\subsubsection{Discrete}
\label{sec:org74bfe64}
\begin{enumerate}
\item State Equation
\label{sec:orga8e43fc}
Need to solve for the matricies in \(\underline{q}[n+1]=\underline{A}q[n]+\underline{b}x[n]\)
\begin{minted}[]{octave}
syms q1n q1n1 q2n q2n1 q3n q3n1 q4n q4n1

eq1 = subs(eq1,zt,q1n);
eq1 = subs(eq1,vt,q2n);
eq1 = subs(eq1,zs,q3n);
eq1 = subs(eq1,vs,q4n);
eq1 = subs(eq1,zt1,q1n1);
eq1 = subs(eq1,vt1,q2n1);
eq1 = subs(eq1,zs1,q3n1);
eq1 = subs(eq1,vs1,q4n1);

eq2 = subs(eq2,zt,q1n);
eq2 = subs(eq2,vt,q2n);
eq2 = subs(eq2,zs,q3n);
eq2 = subs(eq2,vs,q4n);
eq2 = subs(eq2,zt1,q1n1);
eq2 = subs(eq2,vt1,q2n1);
eq2 = subs(eq2,zs1,q3n1);
eq2 = subs(eq2,vs1,q4n1);

equq1n1 = q1n+Ts*q2n; % t
equq3n1 = q3n+Ts*q4n; % s

eq1 = subs(eq1, q1n1, equq1n1);
eq1 = subs(eq1, q3n1, equq3n1);
eq2 = subs(eq2, q1n1, equq1n1);
eq2 = subs(eq2, q3n1, equq3n1);

latex(expand(simplify(solve(eq1,q2n1))))
latex(expand(simplify(solve(eq2,q4n1))))

\end{minted}



Which gives us the following equations:
\begin{itemize}
\item \(q_1[n+1]=q_1[n]+T_s\cdot q_2[n]\)
\item \(q_4[n+1]=-\frac{T_s c_s q_2[n]}{m_t} + \frac{T_s c_s q_4[n]}{m_t} - \frac{T_s k_s q_1[n]}{m_t} + \frac{T_s k_s q_3[n]}{m_t} - \frac{T_s k_t q_1[n]}{m_t} + \frac{T_s k_t u[n]}{m_t} + q_4[n]\)
\item \(q_3[n+1]=q_3[n]+T_s\cdot q_4[n]\)
\item \(q_2[n+1]=\frac{T_s c_s q_2[n]}{m_s} - \frac{T_s c_s q_4[n]}{m_s} + \frac{T_s k_s q_1[n]}{m_s} - \frac{T_s k_s q_3[n]}{m_s} + q_2[n]\)
\end{itemize}


Therefore:
\[\underline{A} = \begin{bmatrix}
1 & T_s & 0 & 0 \\
\frac{T_s k_s q_1[n]}{m_t} +  \frac{T_s kt q_1[n]}{m_t} &\frac{T_s c_s q_2[n]}{m_t} +1 & - \frac{T_s k_s q_3[n]}{m_t} & - \frac{T_s c_s q_4[n]}{m_t} \\
0 & 0 & 1 & T_s \\
-\frac{T_s k_s q_1[n]}{m_s} & -\frac{T_s c_s q_2[n]}{m_s} & \frac{T_s k_s q_3[n]}{m_s} & \frac{T_s c_s q_4[n]}{m_s} +1
\end{bmatrix}\]
\[\underline{B} = \begin{bmatrix}
0 \\
0 \\
0 \\
\frac{T_s k_t}{m_t}
\end{bmatrix}\]
\item Output Equation
\label{sec:org92232c3}
Need to solve for the matricies in \(y[n]=\underline{C}q[n]+d x[n]\)

Using:
\begin{itemize}
\item \(q_1[n] = z_t[n]\)
\item \(q_3[n] = z_s[n]\)
\end{itemize}

Therefore:
\[\underline{C} = \begin{bmatrix}
1 & 0 & 0 & 0 \\
0 & 0 & 1 & 0
\end{bmatrix}\]

\[d = \begin{bmatrix}
0 \\
0
\end{bmatrix}\]
\end{enumerate}
\subsection{d}
\label{sec:orgb78a451}

This code defines the array of systems that will be used
\begin{minted}[]{octave}
clear
clc
pkg load symbolic
pkg load control
% Using student IDs 652137 and 651790
%unsprungMass=236;
%sprungMass=2296;

ms=2296;
mt=236;
kt=250000;
ksMin = 10000;
ksMax = 250000;
csMin = 500;
csMax = 2000;
Ts = 0.1;
t = 0:Ts:1;
idx=1;
% Get an array of systems based on the possible values for ks and cs
numOfSys = 1;
for i =0:numOfSys;
    for j =0:numOfSys;
        % Get the ks and cs for the current system
        ks = ksMin + (i/numOfSys)*(ksMax-ksMin);
        cs = csMin + (i/numOfSys)*(csMax-csMin);

A=[0,1,0,0; ...
    (ks)/(mt)+(kt)/(mt),(cs)/(mt),-(ks)/(mt),-(cs)/(mt);...
    0,0,0,1;...
    -(ks)/(ms),-(cs)/(ms), (ks)/(ms),(cs)/(ms)];
    B=[0 ; ...
       (kt)/(mt) ; ...
       0 ; ...
       0];
        C=[1 , 0 , 0 , 0 ; ...
           0 , 0 , 1 , 0];
        D=[0 ; ...
           0];

        sysArray(idx).A = A;
        sysArray(idx).B = B;
        sysArray(idx).C = C;
        sysArray(idx).D = D;
        idx = idx +1;
    end
end


\end{minted}


Check if the system is stable using the eigenvalues
\begin{minted}[]{octave}
for i = 1:length(sysArray)
  eigen=eig(sysArray(i).A);
  if (all(abs(eigen)) < 1)
    fprintf("The %i th system is stable\n", i)
  else
    fprintf("The %i th system is unstable\n", i)
  end
end
\end{minted}



It appears that all of the systems are unstable. So we will expect the impulse response to get really large as t does.

The impuse response:
\begin{minted}[]{octave}

sys = cell(length(sysArray), 1);
for i = 1:length(sysArray)
    sys{i} = ss(sysArray(i).A, sysArray(i).B, sysArray(i).C, sysArray(i).D);
end

impulseResponses = cell(length(sysArray), 1);

for i = 1:length(sysArray)
    [y, t] = impulse(sys{i});
    impulseResponses{i} = [y,t];
end


figure;
hold on;
for i = 1:length(sysArray)
    plot(impulseResponses{i}(:,3), impulseResponses{i}(:,1), 'DisplayName', sprintf('System %d', i));
    plot(impulseResponses{i}(:,3), impulseResponses{i}(:,2), 'DisplayName', sprintf('System %d', i));
end
hold off;
xlabel('Time (s)');
ylabel('Impulse Response');
title('Impulse Responses of All Systems');
legend show;
grid on;
\end{minted}
\subsection{e}
\label{sec:org3084835}
\begin{minted}[]{octave}
t = 0:Ts:25;
f = 10;
w0 = 2*pi*f;
um = 1;
u = um*sin(w0*t);
[y, t] = lsim(sys, u, t);

figure;
plot(t, y(:, 1), 'b')
hold on;
plot(t, y(:, 2), 'r')
title('Impulse response of the car suspension');
xlabel('Time (s)');
ylabel('Displacement (m)');
legend('Body Displacement', 'Wheel Displacement');
grid on;
\end{minted}

\phantomsection
\label{}
\begin{verbatim}
warning: lsim: arguments number 1 are invalid and are being ignored
error: lsim: require at least one LTI model
error: called from
    lsim at line 94 column 5
warning: opengl_renderer: data values greater than float capacity.  (1) Scale data, or (2) Use gnuplot
warning: called from
    uimenu at line 97 column 8
    __add_default_menu__ at line 68 column 5
    figure at line 97 column 5
error: __plt2vv__: vector lengths must match
error: called from
    __plt__>__plt2vv__ at line 489 column 5
    __plt__>__plt2__ at line 248 column 14
    __plt__ at line 115 column 16
    plot at line 240 column 10
error: __plt2vv__: vector lengths must match
error: called from
    __plt__>__plt2vv__ at line 489 column 5
    __plt__>__plt2__ at line 248 column 14
    __plt__ at line 115 column 16
    plot at line 240 column 10
error: legend: no valid object to label
error: called from
    legend>parse_opts at line 776 column 7
    legend at line 216 column 8
\end{verbatim}
\end{document}
