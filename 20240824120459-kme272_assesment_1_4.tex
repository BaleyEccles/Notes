% Created 2024-08-28 Wed 06:35
% Intended LaTeX compiler: pdflatex
\documentclass[11pt]{article}
\usepackage[utf8]{inputenc}
\usepackage[T1]{fontenc}
\usepackage{graphicx}
\usepackage{longtable}
\usepackage{wrapfig}
\usepackage{rotating}
\usepackage[normalem]{ulem}
\usepackage{amsmath}
\usepackage{amssymb}
\usepackage{capt-of}
\usepackage{hyperref}
\usepackage{minted}
\usepackage[a4paper, margin=2.5cm]{geometry}
\usepackage{minted}
\usepackage{fontspec}
\setmonofont{Iosevka}
\setminted{fontsize=\small, frame=single, breaklines=true}
\usemintedstyle{emacs}
\author{Baley Eccles - 652137}
\date{\textit{{[}2024-08-24 Sat 12:04]}}
\title{KME272 - Assesment 1.4}
\hypersetup{
 pdfauthor={Baley Eccles - 652137},
 pdftitle={KME272 - Assesment 1.4},
 pdfkeywords={},
 pdfsubject={},
 pdfcreator={Emacs 29.4 (Org mode 9.8)}, 
 pdflang={English}}
\begin{document}

\maketitle
\tableofcontents

\section{KME272 - Assesment 1.4}
\label{sec:orgb8ce55f}
\subsection{Q3}
\label{sec:orgc837f5b}

\begin{enumerate}
\item Make an initial guess \((x_0,y_0,z_0,t_0)\)
\item Construct the matrix \(J^{(m)}\)
\item Solve \(J^{(m)}c^{(m)}=-f^{(m)}\) for \(c^{(m)}\)
\item Update the guess \(x^{(m+1)}=x^{(m)}+c^{(m)}\)
\item Check if it has converged \(\lvert \lvert f^{(m)}\rvert \rvert_{2} < 10^{-6}\)
\item If not converged repeat steps 2 to 5
\end{enumerate}
\subsection{Q4}
\label{sec:org37f311a}


\begin{minted}[]{octave}
c = 299792.458;
pos = [ -15093, -519, -13414;
        -5681, 9216, -17053;
        -6228, 16581, -9711;
        -16728, 9532, -6110];
t = [0.069121, 0.071234, 0.070942, 0.070537];
                                % Initial Guess
x = [0; 0; 0; 0];
tol = 10^-6;
i = 0;
maxit = 1000;
while true

                                % Calculate f
  f = zeros(4, 1);
  for k = 1:length(f)
    f(k) = (x(1)-pos(k,1))^2 + ...
           (x(2)-pos(k,2))^2 + ...
           (x(3)-pos(k,3))^2 - ...
           c^2*(x(4)-t(k))^2;
  end
                                % Check for convergence
  if norm(f, 2) < tol
    fprintf('Converged after %i iterations.\n', i);
    break;
  end
                                % Calculate J(m)
  J = zeros(4, 4);
  for k = 1:length(J)
    J(k, :) = [2*(x(1)-pos(k,1)), ...
               2*(x(2)-pos(k,2)), ...
               2*(x(3)-pos(k,3)), ...
               -2*c^2*(x(4)-t(k))];
  end
                                % Solve for c(m)
  c_m = mldivide(J,-f);
                                % Update guess
  x = x + c_m;
  i = i + 1;
  if i > maxit
    fprintf("Max iterations reached (%i)\n", i)
    break;
  end
end
                                % Print results
fprintf("Calculated position and time correction (x, y, z, t): (%.2f,%.2f,%.2f,%.2f)\n", x(1),x(2),x(3),x(4))
\end{minted}

\phantomsection
\label{Q4}
\begin{verbatim}
Converged after 5 iterations.
Calculated position and time correction (x, y, z, t): (3438.33,-3491.41,4071.92,-0.02)
\end{verbatim}


The code was tried with different inital guesses, the number of itterations required for convergence was typically more than with the inital guess being 0. \(x^{(0)}=0\) is a good inital guess becaues if the location of the reciver is close to \((x,y,z)=(0,0,0)\), then the Newton Raphson method will converge quickly.
\subsection{Q5}
\label{sec:org0686e9a}
\begin{minted}[]{octave}
x = [3438.332915,-3491.409159,4071.923288];
d = norm(x);
h = d - 6371;
fprintf("Recivers distance from the surface of the earth (km): %.4f\n", h)
\end{minted}

\phantomsection
\label{Q5}
\begin{verbatim}
Recivers distance from the surface of the earth (km): 0.2346
\end{verbatim}
\end{document}
