% Created 2024-09-26 Thu 16:07
% Intended LaTeX compiler: pdflatex
\documentclass[11pt]{article}
\usepackage[utf8]{inputenc}
\usepackage[T1]{fontenc}
\usepackage{graphicx}
\usepackage{longtable}
\usepackage{wrapfig}
\usepackage{rotating}
\usepackage[normalem]{ulem}
\usepackage{amsmath}
\usepackage{amssymb}
\usepackage{capt-of}
\usepackage{hyperref}
\usepackage{minted}
\usepackage[a4paper, margin=2.5cm]{geometry}
\usepackage{minted}
\usepackage{fontspec}
\setmonofont{Iosevka}
\setminted{fontsize=\small, frame=single, breaklines=true}
\usemintedstyle{emacs}
\author{Baley Eccles - 652137}
\date{\textit{{[}2024-09-17 Tue 10:17]}}
\title{ENG231 - Electrical Machines And Transformers - Assesment 2 Lab 4}
\hypersetup{
 pdfauthor={Baley Eccles - 652137},
 pdftitle={ENG231 - Electrical Machines And Transformers - Assesment 2 Lab 4},
 pdfkeywords={},
 pdfsubject={},
 pdfcreator={Emacs 29.4 (Org mode 9.8)}, 
 pdflang={English}}
\begin{document}

\maketitle
\tableofcontents

\section{ENG231 - Electrical Machines And Transformers - Assesment 2 Lab 4}
\label{sec:orgfc9238e}
\subsection{Name Plate}
\label{sec:org03d8851}
\begin{longtable}{|l|l|}
\hline
VA ratings & 500\\
\hline
\endfirsthead
\multicolumn{2}{l}{Continued from previous page} \\
\hline

VA ratings & 500 \\

\hline
\endhead
\hline\multicolumn{2}{r}{Continued on next page} \\
\endfoot
\endlastfoot
\hline
Primary voltage (\(V\)) & 240\\
\hline
Secondary voltage (\(V\)) & 115\\
\hline
Primary current (\(A\)) & 2.1\\
\hline
Secondary current (\(A\)) & 4.4\\
\hline
Turns ratio & 2.1\\
\hline
\end{longtable}
\subsection{DC Test}
\label{sec:orgb6f6dce}
\begin{longtable}{|l|l|}
\hline
Primary resistance (\(R_{1dc}\)) (\(\Omega\)) & 1.5\\
\hline
\endfirsthead
\multicolumn{2}{l}{Continued from previous page} \\
\hline

Primary resistance (\(R_{1dc}\)) (\(\Omega\)) & 1.5 \\

\hline
\endhead
\hline\multicolumn{2}{r}{Continued on next page} \\
\endfoot
\endlastfoot
\hline
Secondary resistance (\(R_{2dc}\)) (\(\Omega\)) & 0.4\\
\hline
\end{longtable}

The primary side has a higher resistance than the secondary side, this is because the primary side has more turns.
\begin{align*}
a&=2.1 \\
R_{eqHV}&=a^{2}R_2+R_1\\
R_{eqHV}&=3.242\Omega \\
R_{eqLV}&=(1/a)^{2}R_1+R_2\\
R_{eqLV}&=1.12\Omega \\
\end{align*}
\subsection{Open Circuit Test}
\label{sec:orgd55cbf3}
\begin{longtable}{|l|l|l|l|l|l|}
\hline
 & Primary &  &  &  & Secondary\\
\hline
\endfirsthead
\multicolumn{6}{l}{Continued from previous page} \\
\hline

 & Primary &  &  &  & Secondary \\

\hline
\endhead
\hline\multicolumn{6}{r}{Continued on next page} \\
\endfoot
\endlastfoot
\hline
 & V1 & I1 & Poc & PF & V2\\
\hline
LV side open & 110 & 0.563 & 10.3 & 0.165 & 220\\
\hline
HV side open & 240 & 0.375 & 12.5 & 0.138 & 120\\
\hline
\end{longtable}

{[}Group] Calculate turns ratio for your transformer (a = N1 / N2) based on measured open-circuit voltages. Explain why there is a difference (if there is any difference) between your measurements and the nameplate voltages?
\begin{align*}
&\textrm{LV side open} & &\textrm{HV side open} \\
a&=\frac{V_{2}}{V_{1}} & a&=\frac{V_{2}}{V_{1}} \\
a&=2 & a&=2
\end{align*}

{[}Group] Calculate power factor from voltage, current and power measurements and verify that it matches your measured value (from power analyser)
\begin{align*}
&\textrm{LV side open} & &\textrm{HV side open} \\
PF&=\frac{P_{oc}}{V_1I_1} & PF&=\frac{P_{oc}}{V_1I_1} \\
&=0.16631 & &= 0.13889
\end{align*}
The calculated power factor is very close to the measured power factor.

{[}Group] Calculate the core resistance Rc1 and the magnetizing reactance Xm1. (Do this now in the lab and check with the demonstrator that you have something reasonable before you continue)
\begin{align*}
&\textrm{LV side open} & &\textrm{HV side open} \\
R_{c1}&=\frac{V_1^2}{P_{oc}} & R_{c1}&=\frac{V_1^2}{P_{oc}} \\
&=1174.76\Omega & &=4680\Omega \\
X_{m1}&=\frac{V_1}{\sqrt{I_1^2+\left(\frac{V_1}{R_{c1}}\right)^2}} & X_{m1}&=\frac{V_1}{\sqrt{I_1^2+\left(\frac{V_1}{R_{c1}}\right)^2}} \\
&=192.73\Omega & &=633.92\Omega
\end{align*}

{[}Individual] Comment on the differences, if any, between supplying power and measuring from
the HV side or the LV side The LV side shows a higher power compared to the HV side. This means that the LV side consumes more power. This can makes sense with the calculated resistances, the LV side has a higher resistance and hence a higher power draw.

{[}Individual] Calculate the \% increase in current observed when voltage is increased by 20\% above rated voltage? Comment on your observations and discuss?
TODO: I dont think we did this

{[}Individual] On the Power Analyser (still while operating at 20\% above rated voltage) observe transformer supply V and I waveforms. Include a sketch or image of a key waveform observed to help describe what you have observed and why? Does the waveform vary as supply voltage is varied?
\subsection{Short Circus Test}
\label{sec:org4254d38}
\begin{longtable}{|l|l|l|l|l|l|}
\hline
 & Primary &  &  &  & Secondary\\
\hline
\endfirsthead
\multicolumn{6}{l}{Continued from previous page} \\
\hline

 & Primary &  &  &  & Secondary \\

\hline
\endhead
\hline\multicolumn{6}{r}{Continued on next page} \\
\endfoot
\endlastfoot
\hline
 & V1 & I1 & Psc & PF & I2\\
\hline
LV side Short circuited & 7 & 1.76 & 10 & 0.863 & 3.5\\
\hline
\end{longtable}

{[}Group] Calculate \(R_{eq}\) \& \(X_{eq}\).
\begin{align*}
R_{eq}&=\frac{P_{sc}}{I_1^2} \\
&=3.228\Omega \\
X_{eq}&=\sqrt{\left(\frac{V_1}{I_1}\right)^2-R_{eq}^{2}} \\
&=5.123\Omega
\end{align*}

{[}Group] Calculate power factor from these values and verify your measured value.
\begin{align*}
PF&=\frac{P_{sc}}{V_1I_1} \\
&=0.81169
\end{align*}
The calculated power factor matches the measured one.

{[}Individual] Compare \(R_{eq}\) (equivalent winding AC resistance) to the DC resistance values measured earlier (you will need to refer them both to the same side). Why do you think there is a difference (if there is any)?
\[R_{dcTotal}=3.1\Omega\]
The AC resistance is slightly larger than the DC resistance

{[}Individual] Draw the full equivalent circuit for your transformer, labelling impedances with your determined transformer parameters

TODO
\subsection{Performance Test / Full Load Test}
\label{sec:orgcdc0280}
\begin{minted}[]{octave}
clear
clc
close
V2=[ 120, 119, 118, 118, 117, 116, 115];
P1=[ 80, 102, 147, 191, 277, 409, 537];
P2=[ 67, 90, 133, 177, 262, 386, 509];
I2=[ 0.56, 0.753, 1.12, 1.5, 2.2, 3.3, 4.4];
Eff=P2./P1*100;

subplot(2, 1, 1);
plot(I2, V2, 'o', 'LineWidth', 2);
xlabel('Secondary Current (A)');
ylabel('Measured Secondary Voltage (V)');
title('Measured Secondary Voltage vs. Secondary Current');
grid on;

subplot(2, 1, 2);
plot(I2, Eff, 'o', 'LineWidth', 2);
xlabel('Secondary Current (A)');
ylabel('Efficiency');
title('Efficiency vs. Secondary Current');
grid on;

title('Voltage and Efficiency vs. Secondary Current');
\end{minted}

\begin{minted}[]{octave}
clear
clc
close
% Transformer Parameters
V1_rated = 240; % Primary rated voltage (V)
V2_rated = 115; % Secondary rated voltage (V)
R1 = 1.5;       % Primary resistance (Ω)
R2 = 0.4;       % Secondary resistance (Ω)
X1 = 192.73;       % Primary reactance (Ω)
X2 = 633.92;       % Secondary reactance (Ω)
S_rated = 10^3; % Rated power (VA)

% Load Conditions (Ohms)
load_resistances = [200, 150, 100, 75, 50, 33, 25]; % Load values (Ω)

% Initialize results
results = zeros(length(load_resistances), 8); % Columns for V1, I1, P1, PF, V2, I2, P2, Efficiency

for i = 1:length(load_resistances)
    R_load = load_resistances(i);

    % Calculate secondary current (I2)
    I2 = V2_rated / R_load; % Assuming full load voltage

    % Calculate secondary voltage (V2)
    V2 = V2_rated - (I2 * R2) - (I2 * 1j * X2); % Complex voltage

    % Calculate primary current (I1)
    I1 = (I2 * V2_rated) / V1_rated; % Assuming ideal transformer

    % Calculate power (assuming PF = 1 for simplicity)
    P2 = V2 * I2; % Output power
    P1 = V1_rated * I1; % Input power (ideal case)

    % Calculate efficiency
    efficiency = (abs(P2) / abs(P1)) * 100; % Efficiency in percentage

    % Store results
    results(i, :) = [V1_rated, I1, abs(P1), 1, abs(V2), I2, abs(P2), efficiency];
end

% Display results
disp('Results (V1, I1, P1, PF, V2, I2, P2, Efficiency):');
disp(results);

% Plotting
figure;
subplot(2, 1, 1);
plot(load_resistances, results(:, 5), '-o'); % Secondary Voltage
xlabel('Load Resistance (Ω)');
ylabel('Secondary Voltage (V)');
title('Secondary Voltage vs Load Resistance');

subplot(2, 1, 2);
plot(load_resistances, results(:, 8), '-o'); % Efficiency
xlabel('Load Resistance (Ω)');
ylabel('Efficiency (%)');
title('Efficiency vs Load Resistance');

\end{minted}

\begin{minted}[]{octave}
clear
clc
close

% Transformer Parameters
V1_rated = 240; % Primary rated voltage (V)
V2_rated = 115; % Secondary rated voltage (V)
R1 = 1.5;       % Primary resistance (Ω)
R2 = 0.4;       % Secondary resistance (Ω)
X1 = 5.123;    % Primary reactance (Ω)
S_rated = 500;  % Rated power (VA)


\end{minted}
\subsection{Three-phase Transformer Configurations}
\label{sec:org8ccf199}
\subsubsection{Y-Y Connected Transformer}
\label{sec:org88b83c5}
\begin{longtable}{|l|l|l|l|l|l|l|}
\hline
 & Primary Side &  &  &  & Secondary Side & \\
\hline
\endfirsthead
\multicolumn{7}{l}{Continued from previous page} \\
\hline

 & Primary Side &  &  &  & Secondary Side &  \\

\hline
\endhead
\hline\multicolumn{7}{r}{Continued on next page} \\
\endfoot
\endlastfoot
\hline
Quantity & Expected & Observed &  & Quantity & Expected & Observed\\
\hline
VRN & 139 & 139 &  & Vrn & 139 & 139\\
\hline
VWN & 139 & 141 &  & Vwn & 139 & 142\\
\hline
VBN & 139 & 139 &  & Vbn & 139 & 139\\
\hline
VRW & 240 & 243 &  & Vrw & 240 & 243\\
\hline
VWB & 240 & 243 &  & Vwb & 240 & 243\\
VBR & 240 & 240 &  & Vbr & 240 & 240\\
\hline
\end{longtable}
\subsubsection{\(\Delta\)-Y Connected Transformer}
\label{sec:org7fd7304}
\begin{longtable}{|l|l|l|l|l|l|l|}
\hline
 & Primary Side &  &  &  & Secondary Side & \\
\hline
\endfirsthead
\multicolumn{7}{l}{Continued from previous page} \\
\hline

 & Primary Side &  &  &  & Secondary Side &  \\

\hline
\endhead
\hline\multicolumn{7}{r}{Continued on next page} \\
\endfoot
\endlastfoot
\hline
Quantity & Expected & Observed &  & Quantity & Expected & Observed\\
\hline
VRW & 181 & 183 &  & Vrn & 181 & 180\\
\hline
VWB & 181 & 181 &  & Vwn & 181 & 183\\
\hline
VBR & 181 & 181 &  & Vbn & 181 & 181\\
\hline
 &  &  &  & Vrw & 315 & 315\\
\hline
 &  &  &  & Vwb & 315 & 315\\
\hline
 &  &  &  & Vbr & 315 & 320\\
\hline
\end{longtable}
\subsubsection{Y-\(\Delta\) Connected Transformer}
\label{sec:org04f6dc2}
\begin{longtable}{|l|l|l|l|l|l|l|}
\hline
 & Primary Side &  &  &  & Secondary Side & \\
\hline
\endfirsthead
\multicolumn{7}{l}{Continued from previous page} \\
\hline

 & Primary Side &  &  &  & Secondary Side &  \\

\hline
\endhead
\hline\multicolumn{7}{r}{Continued on next page} \\
\endfoot
\endlastfoot
\hline
Quantity & Expected & Observed &  & Quantity & Expected & Observed\\
\hline
VRN & 139 & 141 &  & Vrw & 139 & 141\\
\hline
VWN & 139 & 142 &  & Vwb & 139 & 140\\
\hline
VBN & 139 & 140 &  & Vbr & 139 & 140\\
\hline
VRW & 240 & 245 &  &  &  & \\
\hline
VWB & 240 & 243 &  &  &  & \\
\hline
VBR & 240 & 242 &  &  &  & \\
\hline
\end{longtable}
\subsubsection{\(\Delta\)-\(\Delta\) Connected Transformer}
\label{sec:org8115f67}
\begin{longtable}{|l|l|l|l|l|l|l|}
\hline
 & Primary Side &  &  &  & Secondary Side & \\
\hline
\endfirsthead
\multicolumn{7}{l}{Continued from previous page} \\
\hline

 & Primary Side &  &  &  & Secondary Side &  \\

\hline
\endhead
\hline\multicolumn{7}{r}{Continued on next page} \\
\endfoot
\endlastfoot
\hline
Quantity & Expected & Observed &  & Quantity & Expected & Observed\\
\hline
VRW & 240 & 243 &  & Vrw & 240 & 243\\
\hline
VWB & 240 & 243 &  & Vwb & 240 & 243\\
\hline
VBR & 240 & 240 &  & Vbr & 240 & 240\\
\hline
\end{longtable}
\end{document}
