% Created 2024-08-06 Tue 18:37
% Intended LaTeX compiler: pdflatex
\documentclass[11pt]{article}
\usepackage[utf8]{inputenc}
\usepackage[T1]{fontenc}
\usepackage{graphicx}
\usepackage{longtable}
\usepackage{wrapfig}
\usepackage{rotating}
\usepackage[normalem]{ulem}
\usepackage{amsmath}
\usepackage{amssymb}
\usepackage{capt-of}
\usepackage{hyperref}
\date{\textit{{[}2024-08-02 Fri 21:46]}}
\title{KME272 - Assesment 1.1}
\hypersetup{
 pdfauthor={},
 pdftitle={KME272 - Assesment 1.1},
 pdfkeywords={},
 pdfsubject={},
 pdfcreator={Emacs 29.4 (Org mode 9.8)}, 
 pdflang={English}}
\begin{document}

\maketitle
\tableofcontents

\section{Q1}
\label{sec:org22684f8}
\subsection{(i)}
\label{sec:org9c183ce}
\begin{align*}
& 2x^2-150x =0 \\
& x(2x-150) =0 \\
& x =0 \textrm{ or } 2x-150=0 \\
& x =0 \textrm{ or } x=75
\end{align*}
\subsection{(ii)}
\label{sec:orgd82bf82}
\begin{align*}
x&=\frac{-b \pm \sqrt{b^2-4ac}}{2a} \\
x&=\frac{-150 \pm \sqrt{150^2-4\cdot 2\cdot 3}}{2\cdot 2} \\
x&=74.979 994 66 \textrm{ or } 20.005 336 18 \cdot 10^{−3}
\end{align*}
\subsection{(iii)}
\label{sec:org1cfa61e}
\begin{center}
\begin{tabular}{ |c|c|c|c| }
\hline
 digits of precision & \[b^2-4ac\] & \[\sqrt{b^2-4ac}\] & \[x_2=\frac{-b-\sqrt{b^2-4ac}}{2a}\] \\ \hline
1                    & 40000       & 200                & 0                                    \\ \hline
2                    & 22000       & 150                & 0.0                                  \\ \hline
3                    & 22500       & 150                & 0.00                                 \\ \hline
4                    & 22480       & 149.9              & 0.02500                              \\ \hline
5                    & 22476       & 149.92             & 0.020000                             \\ \hline
6                    & 22476       & 149.920            & 0.0200000                            \\ \hline
\end{tabular}
\end{center}

Calculation for two digits of precision with rounding up:
\begin{center}
\begin{tabular}{ |l|l|l| }
\[=b^2-4ac\]                & \[=\sqrt{b^2-4ac}\]     & \[x_2=\frac{-b-\sqrt{b^2-4ac}}{2a}\] \\
\[=150^2-4\cdot 2\cdot 3 \] & \[=\sqrt{22000}\]       & \[x_2=\frac{150-150}{2\cdot 2}\] \\
\[=22476\]                  & \[=148.324\]            & \[x_2=0\] \\
\[=22000\]                  & \[=150\]                & \[x_2=0.0\] \\
\end{tabular}
\end{center}
\subsection{(iv)}
\label{sec:org38095e2}
Catastrophic occurs when two numbers are cloes to one another and are subtracted, resulting in a small number. So, for (4) to be accurate and (3) inaccurate we must have the value of \(-4ac\) being small relatvie to \(b^2\) and \(b<0\) to get \(-b+\sqrt{b^2-4ac}\approx -b+b\) and \(-b-\sqrt{b^2-4ac}\approx -b-b\approx -2b\). And for (3) to be accurate and (4) inaccurate we must have the same situation, but \(b>0\), as it would lead to \(b-\sqrt{b^2-4ac}\approx b-b\) and \(b+\sqrt{b^2-4ac} \approx b+b \approx 2b\).
\subsection{(v)}
\label{sec:orga5b438b}
\begin{center}
\begin{tabular}{ |c|c|c| }
\hline
 digits of precision & \[x_1=\frac{(-b+\sqrt{b^2-4ac})}{2a}\] & \[x_2=\frac{c}{ax_1}\]  \\ \hline
1                    & 100                                    & 0.02                   \\ \hline
2                    & 75                                     & 0.020                 \\ \hline
3                    & 75.0                                   & 0.0200                 \\ \hline
4                    & 74.97                                  & 0.02001                \\ \hline
5                    & 74.980                                 & 0.020005               \\ \hline
6                    & 74.9800                                & 0.0200053              \\ \hline
\end{tabular}
\end{center}
\end{document}
